\chapter{Potential improvements}\label{chapt:improv}

The interpretations presented in chapter \ref{chapt:results} should be taken with a grain of salt as the analysis was conducted on a single Debian release, tested on a single kernel version, for a specific processor model, and with a single workload. Therefore, it is important to consider that the results may not reflect a universal improvement across different combinations of releases, versions, processor models, and workloads.

To provide a more comprehensive and general statement on whether code collocation effectively reduces ITLB pressure in the kernel, it would be necessary to conduct testing and evaluation across a wider range of scenarios. This would involve implementing a more minimal test setup specifically designed to mitigate additional sources of interference, such as minimizing the impact of other applications during the test run.

Moreover, more sophisticated statistical tools should be used to determine if code collocation is statistically significant. For this purpose, non-normal significance tests can be used. \cite[p. 22-25]{patmc} Additionally, it could be beneficial to explore other functionalities that could potentially improve the data collection stages. For example, investigating if the use of the \textit{PEBS} functionality generates more precise stack calls.

\enlargethispage{\baselineskip}
Another area for improvement is the time-consuming process of creating a linker script for every test run and the subsequent recompilation. One way to overcome this could be to hijacking the \textit{Fine-Grained Kernel Address Space Layout Randomization} (FGKASLR) patch \cite{fgkaslrpatch}. Instead of randomizing the kernel code using the Fisher-Yates algorithm, the FGKASLR functionality could be modified to accept a sorted list, enabling the rearrangement of the kernel code based on that list. The FGKASLR patch extends \textit{Kernel Address Space Layout Randomization} (KASLR) by providing a more sophisticated randomised memory layout by rearranging kernel code on a per-function level granularity at load time. \cite{lwnrandom} \cite{fgkaslrartical} This modification could potentially streamline the process and reduce the time required for recompilation.