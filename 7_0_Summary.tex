\chapter{Summary}\label{chapt:summary}

It was shown that code collocation, in conjunction with the C3 heuristic, can effectively reduce ITLB pressure for a specific workload. However, it is important to note that the findings and recommendations presented here are based on a specific Debian release, kernel version, processor model, and workload. For a more general recommendation on whether code collocation effectively reduces ITLB pressure in the kernel, it is highly recommended to conduct further test runs with diverse configurations.

Moreover, it should be noted that code collocation with the C3 heuristic is just one among several techniques that can potentially reduce ITLB pressure. As described in chapter \ref{chapt:heuristic},  the combination of function-sorting with other techniques, which are part of profile guided optimisation, can further contribute to the reduction of ITLB pressure. A project that tries to do this is for example BOLT \cite{llvm-bolt}.

Additionally, there are alternative techniques available, both with and without the use of PGO, that can effectively reduce ITLB pressure. An Intel white paper \cite{intel_opt_runtime} highlights other possibilities, such as leveraging large pages through the use of the \textit{libhugetlbfs} library. The results in \cite{hfsort} indicate that combining the C3 heuristic with huge pages can lead to an even greater reduction of ITLB pressure.